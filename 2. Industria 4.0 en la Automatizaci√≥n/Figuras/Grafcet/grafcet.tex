\documentclass{article} 
\usepackage{tikz} 
\usepackage{grafcet} % Make sure grafcet.sty is in your TeX path 
\begin{document}

\begin{tikzpicture}[>=stealth]

% Step 1: Injection Molding Completed 
\EtapeInit[0,0]{1} 
\Transition{1}

% Divergence: Sensor selection branch (using DivOU to split into three sensor branches) 
\DivOU{X1}{-5/S3,0/S2,5/S1}

% Branch for Sensor3 (first activation: Desbarbado3 process) 
\SequenceTT[S3]{S3}{2}
% Branch for Sensor2 (second activation: Desbarbado2 process) 
\SequenceTT[S2]{S2}{3}
% Branch for Sensor1 (last activation: sand_deburring1 process) 
\SequenceTT[S1]{S1}{4}

% Convergence of the three branches: using longest branch's transition (T4) as reference 
\ConvOU[-3]{T4}{X1}{C1} 
\Etape[C1]{5} % Step 5: Painting station starts 
\Transition{5}
\DecaleNoeudy[-1]{T5}{T5adj} % Shift transition node upward slightly
\Recept{T5adj}{$20\,s$} % 20-second painting delay

% Assembly Process 
\Etape{6} % Step 6: Assembly station 
\Transition{6}
\Recept{T6}{$30\,s$} % 30-second assembly delay

% Merge of lines into packaging conveyor 
\Etape{7} % Step 7: Final product positioned on unified conveyor 
\Transition{7}

% Packaging Process (Pick & Place) - Divergence to represent the two conveyors: product and box 
\DivOU{X7}{-5/boxConv,5/productConv}

% Branch for box process (boxConv stops when boxSensor is triggered) 
\SequenceTT[boxConv]{boxSensor}{8} % Step 8: Box positioned 
% Branch for product process (productConv stops when productSensor is triggered) 
\SequenceTT[productConv]{productSensor}{9} % Step 9: Product waiting for pick

% Convergence of packaging actions 
\ConvOU[-3]{T8,T9}{C2} 
\Etape[C2]{10} % Step 10: Pick and Place packaging complete
\Transition{10} % Create transition T10

% Loop back to start (to indicate continuous operation) 
\LienRetour[15]{T10}{X1} % Increased offset to avoid overlap

\end{tikzpicture}

\end{document}
